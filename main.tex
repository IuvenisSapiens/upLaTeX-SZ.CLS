% As a sample.tex gernerate by uplatex				%請使用uplatex編譯
% uplatex main 
% uplatex main 
% dvipdfmx main 

%\documentclass[b5paper,背景,水印]{sz}
%\documentclass[b5paper,pdfm,背景]{sz} % 调用糸欄,绘制网格。注:糸欄修改需要精确计算。
%\documentclass[b5paper,背景]{sz} % 不使用糸欄。没有网格。
\documentclass[b5paper,网格]{sz}

%	pdfm 為非必選參數
% 紙張默認為 JIS B5
% 詳見 説明書:https://github.com/Steve-Cheung-emct/Manual-of-SZ.CLS



% 页面调整
%為頭注的使用, 設置文本寬度
\setlength{\textwidth}{491 pt}
%版面寬度(即由版心上邉至底邉的距离)
\setlength{\handurawidth}{618 pt}
%頭注領域の計算,將頭注編號輸出為漢字

% 邉注与正文的间距
\setlength{\marginparsep}{15 pt}

% 设定版面宽
\setlength{\textheight}{392 pt}

% 校订坐标的水平、垂直偏移
\setlength{\voffset}{ 0 mm}
\setlength{\hoffset}{ 0 mm}

% 垂直顶端偏移
\setlength{\topmargin}{-19 pt}

% 尾注的间隔
\setlength{\parskip}{0mm}



% 載入自定義的字體包及設定
\usepackage{stonesettings}

%頭注領域の計算%使頭注標識為漢字
\settochuu \kanjichuu %

\maintitle{脂硯齋重評石頭記}
\subtitle{庚辰本}
\author{清\quad 曹雪芹}
\authorfn{庚辰本}

%%%%%%	自定義的公式 示例

\newcommand\sampleEq{%
  \left(\int_0^\infty \frac{\sin x}{\sqrt{x}}dx\right)^2
  = \sum_{k=0}^\infty \frac{(2k)!}{2^{2k}(k!)^2} \frac{1}{2k+1}
  = \prod_{k=1}^\infty \frac{4k^2}{4k^2-1}
  \neq \frac{\pi}{2015}}

%%%%%%	自定義的公式 示例(結束)


\begin{document}

%%% 此處必須使用系統字體進行初始化,否則會產生蜜汁錯誤


%%%%%% 封面 %%%%%%
\pagecolor{konjou}
\ujlreq
\maketitle


\cleardoublepage%

%%%% 更改主色调
%\pagecolor{kihadairo!60!yamabukiiro!40}
\pagecolor{white}


\pagestyle{my} 
\setcounter{szpage}{3}
\large\tableofcontents
\cleardoublepage

\pagestyle{plain}


%%%%%% 正文 %%%%%%
\LARGE 
\ujlreq
\pagestyle{plain}

% As a sample.tex gernerate by uplatex				%請使用uplatex編譯
% uplatex mysample && ptex2pdf -l -u -ot "-kanji=utf8 " -od "-p B5"  mysample

\chapter{測試}

\par\noindent
\ztxt{1A1}{\color{cyan}%
\kansuji12345\kansuji67890
\kansuji12345\kansuji67890
\kansuji12345\kansuji67890
\kansuji12345\kansuji67890
\kansuji12345\kansuji67890\\
頭注字號{9.13}\rensuji{pt}\\\rensuji{@}11.869\rensuji{pt},
行十字。}
\kansuji12345\kansuji67890
\kansuji12345\kansuji67890
\kansuji12345\kansuji67890
\kansuji12345\kansuji67890
\hspace{1zw}使用\verb+\LARGE+命令調用\dash\\
正文字號{15.521}\rensuji{pt}\rensuji{@}{27.39}\rensuji{pt},
一行\rensuji{31}字。正文與割注換算關係為:

\begin{center}
一個正文字 = 1.62 割注字\\
一個正文字 = 2 行間注字
\endnote{需使用以下命令為行間注設置字號,使之恰好為正文字號的一半。\\ \hspace{2zw}
{\ttfamily $\backslash$rubyfontsetup\{$\backslash$mgfamily
$\backslash$fontsize\{8.5pt\}\{10\}$\backslash$selectfont\} }\\[2mm]
另一個方法,自動模式:不設置字號,只設置字體風格,默認振假名為正文字號的一半。如:\\ \hspace{2zw}
{\ttfamily $\backslash$rubyfontsetup\{$\backslash$mgfamily$\backslash$selectfont\} }}
\end{center}

\par\noindent\warichu*{%
\kansuji12345\kansuji67890
\kansuji12345\kansuji67890
\kansuji12345\kansuji67890
\kansuji12345\kansuji67890
\kansuji12345\kansuji67890
\kansuji123
&%
\kansuji12345\kansuji67890
\kansuji12345\kansuji67890
\kansuji12345\kansuji67890
\kansuji12345\kansuji67890
\kansuji12345\kansuji67890
\kansuji12\\
雙行割注字號{10.043}\rensuji{pt}\rensuji{@}{10.043}\rensuji{pt},
行\rensuji{50}字。&}

\clearpage
\par\noindent{\normalsize
正文 normalsize,字號{9.13}\rensuji{pt}\rensuji{@}{16.434}\rensuji{pt},
行\rensuji{55}字。\\
\kansuji12345\kansuji67890
\kansuji12345\kansuji67890
\kansuji12345\kansuji67890
\kansuji12345\kansuji67890
\kansuji12345\kansuji67890
\kansuji12345\kansuji67890}


\par\noindent{\fontsize{11pt}{12}\selectfont
測試字號{10.043}\rensuji{pt}\rensuji{@}{10.043}\rensuji{pt},
行\rensuji{50}字。\\
\kansuji12345\kansuji67890
\kansuji12345\kansuji67890
\kansuji12345\kansuji67890
\kansuji12345\kansuji67890
\kansuji12345\kansuji67890
\kansuji12345\kansuji67890}


\theendnotes


\setcounter{chapter}{15}
\chapter{賈元春才選鳳藻宮 秦鯨卿夭逝黃泉路}

\rubyfontsetup{\mgfamily\fontsize{8.5pt}{10}\selectfont}  % truely 7.760 pt in real dimen.

\par{}賈璉此時没好意思,
\ztxt{16A7}{\甲{〈庚〉:大觀園用省親事出題,是大関鍵事,方見大手筆行文之立意。\\\hfill{}畸笏。}}
只是訕笑吃酒,說「胡說」二字,「快盛飯来,吃碗子還要往珍大爺那邊去商議事呢。」
鳳姐道:「可是別悞了正事。纔剛老爺叫你作什麼?」
{\leavevmode\kern-1.3zw\hbox{\quad}}%
\己{\warichu*{一段趙嫗討情閒文,却引出通部脈絡。所謂由小及大,譬如登高必自卑之意。細思大觀園一事,&若從如何奉旨起造,又如何分派衆人,從頭細細直冩將来,幾千樣細事,如何能順筆一氣冩清?\\{}又將落於死板拮据之鄕,故只用璉鳳夫妻二人一問一答,上用趙嫗討情作引,&下用蓉薔来說事作收,餘者隨筆順筆略一點染,則耀然洞徹矣。此是避難法。}}
賈璉\ruby[Sg]{{道:「就爲\kenten{省親}。」}}{{\甲{二字醒眼之極,却只如此冩来。}}}\\%
鳳姐忙問道:\甲{\warichu*{「忙」字最要緊,特於鳳姐口中出此字,可&知事関巨要,非同淺細,是此書中正眼矣。}}
「省親的事竟準了不成?」
\甲{\warichu*{問得珍重,&可知是萬\\{}人意外之事。&(脂硯)}}
賈璉笑道:「雖不十分準,也有八分準了。」
\甲{\warichu*{如此故頓一筆,更妙!見得事関&重大,非一語可了者,亦是大篇文\\{}章,抑揚&頓挫之至。}}
鳳姐笑道:「可見當今的隆恩。歷来聽書看戲,古時從未有的。」\\%
\甲{\warichu*{於閨閣中作此語,直與&〈擊壤〉同聲。(脂硯)}}
趙媽\odora{}又接口道:\zmark{16A8}
「可是呢,我也老糊塗了。我聽見上\odora{}下\odora{}吵嚷了這些日子,什麼省親不省親,我也不理論他去;如今又說省親,到底是怎麼個原故?」
\ztxt{16A8}{\甲{趙媽一問是文章家進一步門庭法則。\\[3mm]〈庚〉:自政老生日,用降旨截住,賈母等進朝如此熱鬧,用秦業死岔開,只冩幾個「如何」,將潑天喜事交代完了,緊接黛玉回,璉、鳳閒話,以老嫗勾出省親事来。其千頭萬緒,合榫貫連,無一毫痕跡,如此等,是書多多,不能枚舉。想兄在青峺峰上,經煅煉後,參透重関至恒河沙數。如否,余曰萬不能有此機括,有此筆力,恨不得面問果否。嘆嘆!\\\hfill{}丁亥春。畸笏叟。}}
\ruby[g]{{賈璉道:「如今當今貼體萬人之心,世上至大莫如}}{{\甲{大觀園一篇大文,千頭萬緒,從何處冩起,今故用賈璉夫妻問答之間,閒閒敘出,觀者已省大半。}}}
\ruby[g]{{『孝』字,想来父母児女之性,皆}}{{\甲{後再用蓉、薔二人重一渲染。便省却多少贅瘤筆墨。此是避難法。}}}
是一理,不是貴賤上分別的。當今自爲日夜侍奉太上皇、皇太后,尚不能略盡孝意,因見宮裡嬪妃才人等皆是入宮多年,以致抛離父母音容,豈有不思想之理?在児女思想父母,是分所應當。想父母在家,若只管思念児女,竟不能一見,倘因此成疾致病,甚至死亡,皆由朕躬禁錮,不能使其遂天倫之願,亦大傷天和之事。故啓奏太上皇、皇太后,每月逢二六日期,準其椒房眷屬入宮請安看視。于是太上皇、皇太后大喜,深讚當今至孝純仁,體天格物。因此二位老聖人又下旨意,說椒房眷屬入宮,未免有國體儀制,母女尚不能愜懷。竟大開方便之恩,特降諭諸椒房貴戚,除二六日入宮之恩外,凡有重宇別院之家,可以駐蹕関防之處,不妨啓請內廷鑾輿入其私第,庶可略盡骨肉私情、天倫中之至性。此旨一下,誰不踴躍感戴?現今周貴人父親已在家裡動了工了,修蓋省親別院呢。又有吳貴妃的父親吳天佑家,也往城外\ruby[g]{{踏看地方}}{{\甲{又一樣佈置。}}}去了。這豈非有八九分了?」

\chapter{畫圖測試}


\begin{minipage}<y>[htpb]{80mm}
	\begin{center}
    \begin{tikzpicture}[domain=0:4]
		  \draw[very thin,color=gray] (-0.1,-1.1) grid (3.9,3.9);
  		\draw[->] (-0.2,0) -- (4.2,0) node[right] {$x$};
		  \draw[->] (0,-1.2) -- (0,4.2) node[above] {$f(x)$};
		  \draw[color=red]    plot (\x,\x)             node[right] {$f(x) =x$};
  % \x r 表示弧度
		  \draw[color=blue]   plot (\x,{sin(\x r)})    node[right] {$f(x) = \sin x$};
		  \draw[color=orange] plot (\x,{0.05*exp(\x)}) node[right] {$f(x) = \frac{1}{20} \mathrm e^x$};
		\end{tikzpicture}
	\end{center}
\end{minipage}


\clearpage

\begin{minipage}<t>[htpb][80mm][t]{80mm}
	\begin{center}
		\vspace*{60mm}
    \begin{tikzpicture}[domain=0:4,scale=1,rotate=270]
		  \draw[very thin,color=gray] (-0.1,-1.1) grid (3.9,3.9);
  		\draw[->] (-0.2,0) -- (4.2,0) node[right] {$x$};
		  \draw[->] (0,-1.2) -- (0,4.2) node[above] {$f(x)$};
		  \draw[color=red]    plot (\x,\x)             node[right] {$f(x) =x$};
  % \x r 表示弧度
		  \draw[color=blue]   plot (\x,{sin(\x r)})    node[right] {$f(x) = \sin x$};
		  \draw[color=orange] plot (\x,{0.05*exp(\x)}) node[right] {$f(x) = \frac{1}{20} \mathrm e^x$};
		\end{tikzpicture}
	\end{center}
\end{minipage}

\chapter{公式測試}

\begin{minipage}<y>[htpb]{80mm}
		\vspace*{45mm}
	%\begin{center}
			{\normalsize With normalsize 10 pt in class (truely 9.13\,pt in real dimen):
				\[ \sampleEq \]\par}

			{\Large With Large 14 pt in class (truely 12.782\,pt in real dimen):
				\[ \sampleEq \]\par}

			{\footnotesize With footnotesize 8 pt in class (truely 7.304\,pt in real dimen):
				\[ \sampleEq \]\par}
	%\end{center}
\end{minipage}

\clearpage
\begin{minipage}<t>[htpb]{120mm}
		\vspace*{10mm}
	%\begin{center}
			{\normalsize With normalsize 10 pt in class (truely 9.13\,pt in real dimen):
				\[ \sampleEq \]\par}

			{\Large With Large 14 pt in class (truely 12.782\,pt in real dimen):
				\[ \sampleEq \]\par}

			{\footnotesize With footnotesize 8 pt in class (truely 7.304\,pt in real dimen):
				\[ \sampleEq \]\par}
	%\end{center}
\end{minipage}


\endinput

%%%%%% 自定義的封底
\cleardbpage
\pagestyle{empty}
\watermarkoff

\null\thispagestyle{empty}

\clearpage

\thispagestyle{empty}
\begin{minipage}<y>[htpb]{.99\textheight}
	\begin{center}
  	\vspace{111mm} %奥付のページ上部からの位置
  	\fontsize{11pt}{22pt}\mgfamily
		\begin{tabular}{l}
			\multicolumn{1}{c}{\LARGE\mcfamily\bfseries\makebox[10zw][s]{脂硯齋重評石頭記}}\\[0mm] %%タイトル
				\hline
				%\\[-3mm]
			\hspace{2mm}\makebox[5zw][s]{著 者}\hspace{5mm}%
			清\hspace{1mm}\CID{119}\kern-1mm曹\hskip.5zw雪芹、脂硯齋 \hskip.5zw等\\[0mm]  %%著者
			\hspace{2mm}\makebox[5zw][s]{発 行 日}\hspace{5mm}\today\\[0mm] %%発行日。「(西元\number\year~年\number\month~月\number\day~日)」のところに任意の日付を入れてもいい。
			\hspace{2mm}\makebox[5zw][s]{発 行 者}\hspace{5mm}%
			{子\hskip.5zw 康(SteveCheung)}\\[0mm]  %%発行者
			\hspace{2mm}\makebox[5zw][s]{聯 絡 方 式}\hspace{5mm}%
			{dongfang0571@gmail.com} \hspace{8mm} %\\[0mm] \hspace{65mm}
				\hfil		{\fontsize{10pt}{15pt}\gtfamily{\CID{734}商用禁止;轉載自由(保留署名) }}
			 %\\[-3mm]  %%発行者
				\\\hline
		\end{tabular}
	\end{center}
\end{minipage}

\endinput




\end{document}
