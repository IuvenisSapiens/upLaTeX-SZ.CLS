% As a sample.tex gernerate by uplatex				%請使用uplatex編譯
% uplatex mysample && ptex2pdf -l -u -ot "-kanji=utf8 " -od "-p B5"  mysample

\chapter{測試}

{up\LaTeX}常用命令見 class 説明。




\setcounter{chapter}{15}
\chapter{賈元春才選鳳藻宮 秦鯨卿夭逝黃泉路}

\par{}賈璉此時没好意思,\ztxt{16A7}{\甲{〈庚〉:大觀園用省親事出題,是大関鍵事,方見大手筆行文之立意。\\\hfill{}畸笏。}}\ztxt{16A8}{\甲{趙媽一問是文章家進一步門庭法則。\\[3mm]〈庚〉:自政老生日,用降旨截住,賈母等進朝如此熱鬧,用秦業死岔開,只冩幾個「如何」,將潑天喜事交代完了,緊接黛玉回,璉、鳳閒話,以老嫗勾出省親事来。其千頭萬緒,合榫貫連,無一毫痕跡,如此等,是書多多,不能枚舉。想兄在青峺峰上,經煅煉後,參透重関至恒河沙數。如否,余曰萬不能有此機括,有此筆力,恨不得面問果否。嘆嘆!\\\hfill{}丁亥春。畸笏叟。}}只是訕笑吃酒,說「胡說」二字,「快盛飯来,吃碗子還要往珍大爺那邊去商議事呢。」鳳姐道:「可是別悞了正事。纔剛老爺叫你作什麼?」{\leavevmode\kern-1.3zw\hbox{\quad}}\己{\warichu*{一段趙嫗討情閒文,却引出通部脈絡。所謂由小及大,譬如登高必自卑之意。細思大觀園一事,&若從如何奉旨起造,又如何分派衆人,從頭細細直冩將来,幾千樣細事,如何能順筆一氣冩清?\\{}又將落於死板拮据之鄕,故只用璉鳳夫妻二人一問一答,上用趙嫗討情作引,&下用蓉薔来說事作收,餘者隨筆順筆略一點染,則耀然洞徹矣。此是避難法。}}賈璉\ruby[Sg]{{道:「就爲\kenten{省親}。」}}{{\甲{二字醒眼之極,却只如此冩来。}}}\\{}鳳姐忙問道:\甲{\warichu*{「忙」字最要緊,特於鳳姐口中出此字,可&知事関巨要,非同淺細,是此書中正眼矣。}}「省親的事竟準了不成?」\甲{\warichu*{問得珍重,&可知是萬\\{}人意外之事。&(脂硯)}}賈璉笑道:「雖不十分準,也有八分準了。」\甲{\warichu*{如此故頓一筆,更妙!見得事関&重大,非一語可了者,亦是大篇文\\{}章,抑揚&頓挫之至。}}鳳姐笑道:「可見當今的隆恩。歷来聽書看戲,古時從未有的。」\\\甲{\warichu*{於閨閣中作此語,直與&〈擊壤〉同聲。(脂硯)}}趙媽\odora{}又接口道:\zmark{16A8}「可是呢,我也老糊塗了。我聽見上\odora{}下\odora{}吵嚷了這些日子,什麼省親不省親,我也不理論他去;如今又說省親,到底是怎麼個原故?」賈璉道:「\ruby[g]{{如今當今貼體萬人之心,世上至大莫如}}{{\甲{大觀園一篇大文,千頭萬緒,從何處冩起,今故用賈璉夫妻問答之間,}}}\ruby[g]{{『孝』字,想来父母児女之性,皆是一理,不是貴}}{{\甲{閒閒敘出,觀者已省大半。後再用蓉、薔二人重一渲染。便省却多少贅瘤筆墨。此是避難法。}}}賤上分別的。當今自爲日夜侍奉太上皇、皇太后,尚不能略盡孝意,因見宮裡嬪妃才人等皆是入宮多年,以致抛離父母音容,豈有不思想之理?在児女思想父母,是分所應當。想父母在家,若只管思念児女,竟不能一見,倘因此成疾致病,甚至死亡,皆由朕躬禁錮,不能使其遂天倫之願,亦大傷天和之事。故啓奏太上皇、皇太后,每月逢二六日期,準其椒房眷屬入宮請安看視。于是太上皇、皇太后大喜,深讚當今至孝純仁,體天格物。因此二位老聖人又下旨意,說椒房眷屬入宮,未免有國體儀制,母女尚不能愜懷。竟大開方便之恩,特降諭諸椒房貴戚,除二六日入宮之恩外,凡有重宇別院之家,可以駐蹕関防之處,不妨啓請內廷鑾輿入其私第,庶可略盡骨肉私情、天倫中之至性。此旨一下,誰不踴躍感戴?現今周貴人父親已在家裡動了工了,修蓋省親別院呢。又有吳貴妃的父親吳天佑家,也往城外\ruby[g]{{踏看地方}}{{\甲{又一樣佈置。}}}去了。這豈非有八九分了?」

\endinput